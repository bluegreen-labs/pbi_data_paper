% Options for packages loaded elsewhere
\PassOptionsToPackage{unicode}{hyperref}
\PassOptionsToPackage{hyphens}{url}
%
\documentclass[
  landscape]{article}
\usepackage{lmodern}
\usepackage{amssymb,amsmath}
\usepackage{ifxetex,ifluatex}
\ifnum 0\ifxetex 1\fi\ifluatex 1\fi=0 % if pdftex
  \usepackage[T1]{fontenc}
  \usepackage[utf8]{inputenc}
  \usepackage{textcomp} % provide euro and other symbols
\else % if luatex or xetex
  \usepackage{unicode-math}
  \defaultfontfeatures{Scale=MatchLowercase}
  \defaultfontfeatures[\rmfamily]{Ligatures=TeX,Scale=1}
\fi
% Use upquote if available, for straight quotes in verbatim environments
\IfFileExists{upquote.sty}{\usepackage{upquote}}{}
\IfFileExists{microtype.sty}{% use microtype if available
  \usepackage[]{microtype}
  \UseMicrotypeSet[protrusion]{basicmath} % disable protrusion for tt fonts
}{}
\makeatletter
\@ifundefined{KOMAClassName}{% if non-KOMA class
  \IfFileExists{parskip.sty}{%
    \usepackage{parskip}
  }{% else
    \setlength{\parindent}{0pt}
    \setlength{\parskip}{6pt plus 2pt minus 1pt}}
}{% if KOMA class
  \KOMAoptions{parskip=half}}
\makeatother
\usepackage{xcolor}
\IfFileExists{xurl.sty}{\usepackage{xurl}}{} % add URL line breaks if available
\IfFileExists{bookmark.sty}{\usepackage{bookmark}}{\usepackage{hyperref}}
\hypersetup{
  hidelinks,
  pdfcreator={LaTeX via pandoc}}
\urlstyle{same} % disable monospaced font for URLs
\usepackage[margin=1in]{geometry}
\usepackage{graphicx,grffile}
\makeatletter
\def\maxwidth{\ifdim\Gin@nat@width>\linewidth\linewidth\else\Gin@nat@width\fi}
\def\maxheight{\ifdim\Gin@nat@height>\textheight\textheight\else\Gin@nat@height\fi}
\makeatother
% Scale images if necessary, so that they will not overflow the page
% margins by default, and it is still possible to overwrite the defaults
% using explicit options in \includegraphics[width, height, ...]{}
\setkeys{Gin}{width=\maxwidth,height=\maxheight,keepaspectratio}
% Set default figure placement to htbp
\makeatletter
\def\fps@figure{htbp}
\makeatother
\setlength{\emergencystretch}{3em} % prevent overfull lines
\providecommand{\tightlist}{%
  \setlength{\itemsep}{0pt}\setlength{\parskip}{0pt}}
\setcounter{secnumdepth}{-\maxdimen} % remove section numbering
\usepackage{setspace}
\usepackage{lineno}
\usepackage[width=\textwidth]{caption}
\usepackage{booktabs}
\usepackage{longtable}
\usepackage{array}
\usepackage{multirow}
\usepackage{wrapfig}
\usepackage{float}
\usepackage{colortbl}
\usepackage{pdflscape}
\usepackage{tabu}
\usepackage{threeparttable}
\usepackage{threeparttablex}
\usepackage[normalem]{ulem}
\usepackage{makecell}

\author{}
\date{\vspace{-2.5em}}

\begin{document}

\hypertarget{appendix}{%
\section{Appendix}\label{appendix}}

\hypertarget{data-descriptors}{%
\subsection{Data descriptors}\label{data-descriptors}}

\textbackslash begin\{longtable\}{[}t{]}\{ll\}
\textbackslash caption\{\label{tab:unnamed-chunk-1}Master database
column entries (pbi\_master\_image\_stats.csv). We report the label used
as provided in the dataset (csv file) and a brief description of the
data represented in the table column.\}\textbackslash{} \toprule label
\& description\textbackslash{} \midrule \endfirsthead
\textbackslash caption{[}{]}\{Master database column entries
(pbi\_master\_image\_stats.csv). We report the label used as provided in
the dataset (csv file) and a brief description of the data represented
in the table column. \textit{(continued)}\}\textbackslash{} \toprule
label \& description\textbackslash{} \midrule \endhead

\endfoot
\bottomrule
\endlastfoot

report\_id \& unique image based identifier (assigned per
season)\textbackslash{} farmer \& unique farmer id
number\textbackslash{} project\_id \& project id number (1 fand 2 for
the first and second growing season respectively)\textbackslash{} field
\& field id number (when multiple fields are acquired by a
farmer)\textbackslash{} site \& unique site id\textbackslash{}
\addlinespace season\_id \& season id -- defined as
year\_month-of-first-sowing\textbackslash{} lat \& decimal
latitude\textbackslash{} lon \& decimal longitude\textbackslash{} date
\& date of image acquisition (YYYY-MM-DD)\textbackslash{} time \& time
of image acquisition (HH:MM:SS)\textbackslash{} \addlinespace image \&
filename of the image as provided in the provided image
dataset\textbackslash{} r\_dn \& red digital number\textbackslash{}
g\_dn \& green digital number\textbackslash{} b\_dn \& blue digital
number\textbackslash{} rcc\_90 \& 90th percentile of the red chromatic
coordinate\textbackslash{} \addlinespace gcc\_90 \& 90th percentile of
the green chromatic coordinate\textbackslash{} smooth\_gcc\_90 \&
smoothed and normalized estimates of field based Gcc
values\textbackslash{} grvi\_10 \& 10th percentile of the green red
vegetation index\textbackslash{} spatial\_location \& the spatial
location, either a town name or a grid cell index number\textbackslash{}
spatial\_unit \& the spatial aggregation used, either\textbackslash{}
\addlinespace growth\_stage \& wheat growth stage as assigned by human
experts\textbackslash{} lodging\_labels \& lodging labels as assigned by
human experts\textbackslash{} seed\_variety \& farmer reported seed
variety\textbackslash{} dam\_rain \& farmer reported rain
damage\textbackslash{} dam\_hail \& farmer reported hail
damage\textbackslash{} \addlinespace dam\_high\_temp \& farmer reported
temperature damage (heat)\textbackslash{} dam\_low\_temp \& farmer
reported temperature damage (cold)\textbackslash{} dam\_pest \& farmer
reported pest\textbackslash{} dam\_lodging \& farmer reported
lodging\textbackslash{} dam\_wildlife \& farmer reported wildlife
damages\textbackslash{} \addlinespace dam\_fire \& farmer reported fire
damages\textbackslash{} dam\_unclassified \& farmer reported damages,
unclassified\textbackslash{} man\_harvest \& farmer reported management,
harvest\textbackslash{} man\_irrigate \& farmer reported
irrigation\textbackslash{} man\_till \& farmer reported
tilling\textbackslash{} \addlinespace man\_sow \& farmer reported
sowing\textbackslash{} man\_weed \& farmer reported
weeding\textbackslash{} man\_urea\_kg\_acre \& farmer reported
application of urea (kg / acre)\textbackslash{} man\_dap\_kg\_acre \&
farmer reported application of dap (kg / acre)\textbackslash{}
man\_potash\_kg\_acre \& farmer reported application of potash (kg /
acre)\textbackslash{} \addlinespace man\_zinc\_kg\_acre \& farmer
reported application of zinc (kg / acre)\textbackslash{}
man\_fungicide\_kg\_acre \& farmer reported application of fungicide (kg
/ acre)\textbackslash{} man\_herbicide\_kg\_acre \& farmer reported
application of herbicide (kg / acre)\textbackslash{}
man\_pesticide\_kg\_acre \& farmer reported application of pesticide (kg
/ acre)\textbackslash{} man\_unclassified \& unclassified management
intervention\textbackslash{} \addlinespace soil\_type \& soil type
(loam,clay loam,loam sandy,clay or sandy)\textbackslash{} drainage \&
soil drainage (good or poor)\textbackslash{} sowing\_date \& date of
sowing (YYYY-MM-DD)\textbackslash{} harvest\_quantity \& harvest
quantity from crop cutting ()\textbackslash{} yield\_expectation \&
farmer estimated yield at time of assessment ()\textbackslash{}
\addlinespace duration \& duration of the growing season (based upon
acquired image data)\textbackslash{} nr\_values \& nr of processed
images\textbackslash{} spread \& mean time difference between
images\textbackslash{} qa \& ratio of the nr of values (images) and
their spread\textbackslash* \textbackslash end\{longtable\}

\textbackslash begin\{longtable\}{[}t{]}\{ll\}
\textbackslash caption\{\label{tab:unnamed-chunk-2}Derivated summary
database (pbi\_summary\_stats.csv) column entries. We report the label
used as provided in the summary dataset (csv file) and a brief
description of the data represented in the table
column.\}\textbackslash{} \toprule label \& description\textbackslash{}
\midrule \endfirsthead \textbackslash caption{[}{]}\{Derivated summary
database (pbi\_summary\_stats.csv) column entries. We report the label
used as provided in the summary dataset (csv file) and a brief
description of the data represented in the table column.
\textit{(continued)}\}\textbackslash{} \toprule label \&
description\textbackslash{} \midrule \endhead

\endfoot
\bottomrule
\endlastfoot

project\_id \& project id number (1 fand 2 for the first and second
growing season respectively)\textbackslash{} spatial\_unit \& the
spatial aggregation used, either\textbackslash{} spatial\_location \&
the spatial location, either a town name or a grid cell index
number\textbackslash{} lat \& decimal latitude\textbackslash{} lon \&
decimal longitude\textbackslash{} \addlinespace nr\_fields \& number of
fields in a region\textbackslash{} nr\_farmers \& number of farmers in a
region\textbackslash{} man\_mean\_date \& mean manipulation date (across
all interventions)\textbackslash{} man\_sd\_date \& standard deviation
of the manipulation date (across all interventions)\textbackslash{}
nr\_fields\_irrigated \& number of fields irrigated\textbackslash{}
\addlinespace nr\_fields\_weeded \& number of fields
weeded\textbackslash{} nr\_fields\_tilled \& number of fields
tilled\textbackslash{} nr\_fields\_sowed \& number of fields
sowed\textbackslash{} nr\_fields\_harvested \& number of fields
harvested\textbackslash{} nr\_fields\_urea \& number of fields on which
urea was applied\textbackslash{} \addlinespace nr\_fields\_dap \& number
of fields on which dap was applied\textbackslash{} nr\_fields\_potash \&
number of fields on which potash was applied\textbackslash{}
nr\_fields\_zinc \& number of fields on which zinc was
applied\textbackslash{} nr\_fields\_fungicide \& number of fields on
which fungicide was applied\textbackslash{} nr\_fields\_herbicide \&
number of fields on which herbicide was applied\textbackslash{}
\addlinespace nr\_fields\_pesticide \& number of fields on which
pesticide was applied\textbackslash{} mean\_urea\_kg\_acre \& mean urea
applied across the region (kg / acre)\textbackslash{} sd\_urea\_kg\_acre
\& standard deviation on the mean urea applied across the region (kg /
acre)\textbackslash{} mean\_dap\_kg\_acre \& mean urea applied across
the region (kg / acre)\textbackslash{} sd\_dap\_kg\_acre \& standard
deviation on the mean urea applied across the region (kg /
acre)\textbackslash{} \addlinespace mean\_potash\_kg\_acre \& mean urea
applied across the region (kg / acre)\textbackslash{}
sd\_potash\_kg\_acre \& standard deviation on the mean urea applied
across the region (kg / acre)\textbackslash{} mean\_zinc\_kg\_acre \&
mean urea applied across the region (kg / acre)\textbackslash{}
sd\_zinc\_kg\_acre \& standard deviation on the mean urea applied across
the region (kg / acre)\textbackslash{} mean\_fungicide\_kg\_acre \& mean
urea applied across the region (kg / acre)\textbackslash{} \addlinespace
sd\_fungicide\_kg\_acre \& standard deviation on the mean urea applied
across the region (kg / acre)\textbackslash{} mean\_herbicide\_kg\_acre
\& mean urea applied across the region (kg / acre)\textbackslash{}
sd\_herbicide\_kg\_acre \& standard deviation on the mean urea applied
across the region (kg / acre)\textbackslash{} dam\_mean\_date \& mean
date of sustained damage (across all classes)\textbackslash{}
dam\_sd\_date \& standard deviation on the mean date of sustained damage
(across all classes, in days)\textbackslash{} \addlinespace
nr\_fields\_dam\_rain \& number of fields damaged by
rain\textbackslash{} nr\_fields\_dam\_hail \& number of fields damaged
by hail\textbackslash{} nr\_fields\_dam\_high\_temp \& number of fields
damaged by high temperature\textbackslash{} nr\_fields\_dam\_low\_temp
\& number of fields damaged by low temperature\textbackslash{}
nr\_fields\_dam\_pest \& number of fields damaged by
pests\textbackslash{} \addlinespace nr\_fields\_dam\_lodging \& number
of fields damaged by lodging\textbackslash{} nr\_fields\_dam\_wildlife
\& number of fields damaged by wildlife\textbackslash{}
nr\_fields\_dam\_fire \& number of fields damaged by
fire\textbackslash{} nr\_fields\_dam\_unclassified \& number of fields
with unclassified damage\textbackslash{} mean\_qa \& mean quality
assurance value\textbackslash{} \addlinespace mean\_spread \& mean time
difference between images across a region\textbackslash{}
mean\_nr\_values \& mean nr of processed images per field across a
region\textbackslash{} total\_nr\_values \& mean total number of images
per field across a region\textbackslash{} mean\_duration \& mean
duration of the season per field across a region (in
days)\textbackslash{} mean\_sowing\_date \& mean sowing date by field
across a region (YYYY-MM-DD)\textbackslash{} \addlinespace
sd\_sowing\_date \& standard deviation of the sowing date by field
across a region (in days)\textbackslash{} rising \& phenophase of the
rising part of greenness curve (73\% threshold, tillering phase, in
YYYY-MM-DD)\textbackslash{} falling \& phenophase of the falling part of
greenness curve (83\% threshold, heading phase, in
YYYY-MM-DD)\textbackslash{} rising\_lower\_ci \& phenophase confidence
intervals (YYYY-MM-DD)\textbackslash{} rising\_upper\_ci \& phenophase
confidence intervals (YYYY-MM-DD)\textbackslash{} \addlinespace
falling\_lower\_ci \& phenophase confidence intervals
(YYYY-MM-DD)\textbackslash{} falling\_upper\_ci \& phenophase confidence
intervals (YYYY-MM-DD)\textbackslash* \textbackslash end\{longtable\}

\textbackslash begin\{longtable\}{[}t{]}\{ll\}
\textbackslash caption\{\label{tab:unnamed-chunk-3}Derivated time series
summary (pbi\_summary\_time\_series.csv) database column entries. We
report the label used as provided in the time series summary dataset
(csv file) and a brief description of the data represented in the table
column. The time series summary data can be merged with the summary
dataset for quality control.\}\textbackslash{} \toprule label \&
description\textbackslash{} \midrule \endfirsthead
\textbackslash caption{[}{]}\{Derivated time series summary
(pbi\_summary\_time\_series.csv) database column entries. We report the
label used as provided in the time series summary dataset (csv file) and
a brief description of the data represented in the table column. The
time series summary data can be merged with the summary dataset for
quality control. \textit{(continued)}\}\textbackslash{} \toprule label
\& description\textbackslash{} \midrule \endhead

\endfoot
\bottomrule
\endlastfoot

project\_id \& project id number (1 fand 2 for the first and second
growing season respectively)\textbackslash{} spatial\_unit \& the
spatial aggregation used, either GADM or WorldClim 2.5 or 5 degree
units\textbackslash{} spatial\_location \& the spatial location, either
a town name or a grid cell index number\textbackslash{} lat \& decimal
latitude\textbackslash{} lon \& decimal longitude\textbackslash{}
\addlinespace date \& date of reported Gcc values
(YYYY-MM-DD)\textbackslash{} smooth\_gcc\_90 \& Normalized and smoothed
Gcc for a given spatial unit and spatial location\textbackslash{}
smooth\_gcc\_90\_lower\_ci \& Lower confidence interval of the
normalized and smoothed Gcc\textbackslash{} smooth\_gcc\_90\_upper\_ci
\& Upper confidence interval of the normalized and smoothed
Gcc\textbackslash{} rising \& transition date (at 73\% threshold) of the
rising part of the normalized and smoothed Gcc curve (per spatial
location)\textbackslash{} \addlinespace falling \& transition date (at
83\% threshold) of the falling part of the normalized and smoothed Gcc
curve (per spatial location)\textbackslash{} rising\_lower\_ci \&
uncertainty intervals on the rising transition date, based upon the
lower CI\textbackslash{} rising\_upper\_ci \& uncertainty intervals on
the rising transition date, based upon the upper CI\textbackslash{}
falling\_lower\_ci \& uncertainty intervals on the falling transition
date, based upon the lower CI\textbackslash{} falling\_upper\_ci \&
uncertainty intervals on the falling transition date, based upon the
upper CI\textbackslash* \textbackslash end\{longtable\}

\end{document}
